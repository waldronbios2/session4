% Options for packages loaded elsewhere
\PassOptionsToPackage{unicode}{hyperref}
\PassOptionsToPackage{hyphens}{url}
%
\documentclass[
  ignorenonframetext,
]{beamer}
\usepackage{pgfpages}
\setbeamertemplate{caption}[numbered]
\setbeamertemplate{caption label separator}{: }
\setbeamercolor{caption name}{fg=normal text.fg}
\beamertemplatenavigationsymbolsempty
% Prevent slide breaks in the middle of a paragraph
\widowpenalties 1 10000
\raggedbottom
\setbeamertemplate{part page}{
  \centering
  \begin{beamercolorbox}[sep=16pt,center]{part title}
    \usebeamerfont{part title}\insertpart\par
  \end{beamercolorbox}
}
\setbeamertemplate{section page}{
  \centering
  \begin{beamercolorbox}[sep=12pt,center]{part title}
    \usebeamerfont{section title}\insertsection\par
  \end{beamercolorbox}
}
\setbeamertemplate{subsection page}{
  \centering
  \begin{beamercolorbox}[sep=8pt,center]{part title}
    \usebeamerfont{subsection title}\insertsubsection\par
  \end{beamercolorbox}
}
\AtBeginPart{
  \frame{\partpage}
}
\AtBeginSection{
  \ifbibliography
  \else
    \frame{\sectionpage}
  \fi
}
\AtBeginSubsection{
  \frame{\subsectionpage}
}
\usepackage{lmodern}
\usepackage{amssymb,amsmath}
\usepackage{ifxetex,ifluatex}
\ifnum 0\ifxetex 1\fi\ifluatex 1\fi=0 % if pdftex
  \usepackage[T1]{fontenc}
  \usepackage[utf8]{inputenc}
  \usepackage{textcomp} % provide euro and other symbols
\else % if luatex or xetex
  \usepackage{unicode-math}
  \defaultfontfeatures{Scale=MatchLowercase}
  \defaultfontfeatures[\rmfamily]{Ligatures=TeX,Scale=1}
\fi
\usetheme[]{Hannover}
\usecolortheme{dove}
\usefonttheme{structurebold}
% Use upquote if available, for straight quotes in verbatim environments
\IfFileExists{upquote.sty}{\usepackage{upquote}}{}
\IfFileExists{microtype.sty}{% use microtype if available
  \usepackage[]{microtype}
  \UseMicrotypeSet[protrusion]{basicmath} % disable protrusion for tt fonts
}{}
\makeatletter
\@ifundefined{KOMAClassName}{% if non-KOMA class
  \IfFileExists{parskip.sty}{%
    \usepackage{parskip}
  }{% else
    \setlength{\parindent}{0pt}
    \setlength{\parskip}{6pt plus 2pt minus 1pt}}
}{% if KOMA class
  \KOMAoptions{parskip=half}}
\makeatother
\usepackage{xcolor}
\IfFileExists{xurl.sty}{\usepackage{xurl}}{} % add URL line breaks if available
\IfFileExists{bookmark.sty}{\usepackage{bookmark}}{\usepackage{hyperref}}
\hypersetup{
  pdftitle={Session 4: loglinear regression part 1},
  pdfauthor={Levi Waldron},
  hidelinks,
  pdfcreator={LaTeX via pandoc}}
\urlstyle{same} % disable monospaced font for URLs
\newif\ifbibliography
\usepackage{color}
\usepackage{fancyvrb}
\newcommand{\VerbBar}{|}
\newcommand{\VERB}{\Verb[commandchars=\\\{\}]}
\DefineVerbatimEnvironment{Highlighting}{Verbatim}{commandchars=\\\{\}}
% Add ',fontsize=\small' for more characters per line
\usepackage{framed}
\definecolor{shadecolor}{RGB}{248,248,248}
\newenvironment{Shaded}{\begin{snugshade}}{\end{snugshade}}
\newcommand{\AlertTok}[1]{\textcolor[rgb]{0.94,0.16,0.16}{#1}}
\newcommand{\AnnotationTok}[1]{\textcolor[rgb]{0.56,0.35,0.01}{\textbf{\textit{#1}}}}
\newcommand{\AttributeTok}[1]{\textcolor[rgb]{0.77,0.63,0.00}{#1}}
\newcommand{\BaseNTok}[1]{\textcolor[rgb]{0.00,0.00,0.81}{#1}}
\newcommand{\BuiltInTok}[1]{#1}
\newcommand{\CharTok}[1]{\textcolor[rgb]{0.31,0.60,0.02}{#1}}
\newcommand{\CommentTok}[1]{\textcolor[rgb]{0.56,0.35,0.01}{\textit{#1}}}
\newcommand{\CommentVarTok}[1]{\textcolor[rgb]{0.56,0.35,0.01}{\textbf{\textit{#1}}}}
\newcommand{\ConstantTok}[1]{\textcolor[rgb]{0.00,0.00,0.00}{#1}}
\newcommand{\ControlFlowTok}[1]{\textcolor[rgb]{0.13,0.29,0.53}{\textbf{#1}}}
\newcommand{\DataTypeTok}[1]{\textcolor[rgb]{0.13,0.29,0.53}{#1}}
\newcommand{\DecValTok}[1]{\textcolor[rgb]{0.00,0.00,0.81}{#1}}
\newcommand{\DocumentationTok}[1]{\textcolor[rgb]{0.56,0.35,0.01}{\textbf{\textit{#1}}}}
\newcommand{\ErrorTok}[1]{\textcolor[rgb]{0.64,0.00,0.00}{\textbf{#1}}}
\newcommand{\ExtensionTok}[1]{#1}
\newcommand{\FloatTok}[1]{\textcolor[rgb]{0.00,0.00,0.81}{#1}}
\newcommand{\FunctionTok}[1]{\textcolor[rgb]{0.00,0.00,0.00}{#1}}
\newcommand{\ImportTok}[1]{#1}
\newcommand{\InformationTok}[1]{\textcolor[rgb]{0.56,0.35,0.01}{\textbf{\textit{#1}}}}
\newcommand{\KeywordTok}[1]{\textcolor[rgb]{0.13,0.29,0.53}{\textbf{#1}}}
\newcommand{\NormalTok}[1]{#1}
\newcommand{\OperatorTok}[1]{\textcolor[rgb]{0.81,0.36,0.00}{\textbf{#1}}}
\newcommand{\OtherTok}[1]{\textcolor[rgb]{0.56,0.35,0.01}{#1}}
\newcommand{\PreprocessorTok}[1]{\textcolor[rgb]{0.56,0.35,0.01}{\textit{#1}}}
\newcommand{\RegionMarkerTok}[1]{#1}
\newcommand{\SpecialCharTok}[1]{\textcolor[rgb]{0.00,0.00,0.00}{#1}}
\newcommand{\SpecialStringTok}[1]{\textcolor[rgb]{0.31,0.60,0.02}{#1}}
\newcommand{\StringTok}[1]{\textcolor[rgb]{0.31,0.60,0.02}{#1}}
\newcommand{\VariableTok}[1]{\textcolor[rgb]{0.00,0.00,0.00}{#1}}
\newcommand{\VerbatimStringTok}[1]{\textcolor[rgb]{0.31,0.60,0.02}{#1}}
\newcommand{\WarningTok}[1]{\textcolor[rgb]{0.56,0.35,0.01}{\textbf{\textit{#1}}}}
\usepackage{graphicx,grffile}
\makeatletter
\def\maxwidth{\ifdim\Gin@nat@width>\linewidth\linewidth\else\Gin@nat@width\fi}
\def\maxheight{\ifdim\Gin@nat@height>\textheight\textheight\else\Gin@nat@height\fi}
\makeatother
% Scale images if necessary, so that they will not overflow the page
% margins by default, and it is still possible to overwrite the defaults
% using explicit options in \includegraphics[width, height, ...]{}
\setkeys{Gin}{width=\maxwidth,height=\maxheight,keepaspectratio}
% Set default figure placement to htbp
\makeatletter
\def\fps@figure{htbp}
\makeatother
\setlength{\emergencystretch}{3em} % prevent overfull lines
\providecommand{\tightlist}{%
  \setlength{\itemsep}{0pt}\setlength{\parskip}{0pt}}
\setcounter{secnumdepth}{-\maxdimen} % remove section numbering

\title{Session 4: loglinear regression part 1}
\author{Levi Waldron}
\date{}
\institute{CUNY SPH Biostatistics 2}

\begin{document}
\frame{\titlepage}

\hypertarget{learning-objectives-and-outline}{%
\section{Learning objectives and
outline}\label{learning-objectives-and-outline}}

\begin{frame}{Learning objectives}
\protect\hypertarget{learning-objectives}{}

\begin{enumerate}
\tightlist
\item
  Define log-linear models in GLM framework
\item
  Identify situations that motivate use of log-linear models
\item
  Define the Poisson distribution and the log-linear Poisson GLM
\item
  Identify applications and properties of the Poisson distribution
\item
  Define multicollinearity and identify resulting issues
\end{enumerate}

\end{frame}

\begin{frame}{Outline}
\protect\hypertarget{outline}{}

\begin{enumerate}
\tightlist
\item
  Brief review of GLMs
\item
  Motivating example for log-linear models
\item
  Poisson log-linear GLM
\item
  Notes on Multicollinearity
\end{enumerate}

Reading: Vittinghoff textbook chapter 8.1-8.3

\end{frame}

\hypertarget{brief-review-of-glms}{%
\section{Brief review of GLMs}\label{brief-review-of-glms}}

\begin{frame}{Components of GLM}
\protect\hypertarget{components-of-glm}{}

\begin{itemize}
\tightlist
\item
  \textbf{Random component} specifies the conditional distribution for
  the response variable - it doesn't have to be normal but can be any
  distribution that belongs to the ``exponential'' family of
  distributions
\item
  \textbf{Systematic component} specifies linear function of predictors
  (linear predictor)
\item
  \textbf{Link} {[}denoted by g(.){]} specifies the relationship between
  the expected value of the random component and the systematic
  component, can be linear or nonlinear
\end{itemize}

\end{frame}

\begin{frame}{Linear Regression as GLM}
\protect\hypertarget{linear-regression-as-glm}{}

\begin{itemize}
\item
  \textbf{The model}:
  \(y_i = E[y|x] + \epsilon_i = \beta_0 + \beta_1 x_{1i} + \beta_2 x_{2i} + ... + \beta_p x_{pi} + \epsilon_i\)
\item
  \textbf{Random component} of \(y_i\) is normally distributed:
  \(\epsilon_i \stackrel{iid}{\sim} N(0, \sigma_\epsilon^2)\)
\item
  \textbf{Systematic component} (linear predictor):
  \(\beta_0 + \beta_1 x_{1i} + \beta_2 x_{2i} + ... + \beta_p x_{pi}\)
\item
  \textbf{Link function} here is the \emph{identity link}:
  \(g(E(y | x)) = E(y | x)\). We are modeling the mean directly, no
  transformation.
\end{itemize}

\end{frame}

\begin{frame}{Logistic Regression as GLM}
\protect\hypertarget{logistic-regression-as-glm}{}

\begin{itemize}
\item
  \textbf{The model}: \[
  Logit(P(x)) = log \left( \frac{P(x)}{1-P(x)} \right) = \beta_0 + \beta_1 x_{1i} + \beta_2 x_{2i} + ... + \beta_p x_{pi}
  \]
\item
  \textbf{Random component}: \(y_i\) follows a Binomial distribution
  (outcome is a binary variable)
\item
  \textbf{Systematic component}: linear predictor \[
  \beta_0 + \beta_1 x_{1i} + \beta_2 x_{2i} + ... + \beta_p x_{pi}
  \]
\item
  \textbf{Link function}: \emph{logit} (Converts Prob -\textgreater{}
  log-odds) \[
  g(P(x)) = logit(P(x)) = log\left( \frac{P(x)}{1-P(x)} \right)
  \] \[
  P(x) = g^{-1}\left( \beta_0 + \beta_1 x_{1i} + \beta_2 x_{2i} + ... + \beta_p x_{pi}
   \right)
  \]
\end{itemize}

\end{frame}

\begin{frame}{Additive vs.~Multiplicative models}
\protect\hypertarget{additive-vs.-multiplicative-models}{}

\begin{itemize}
\tightlist
\item
  Linear regression is an \emph{additive} model

  \begin{itemize}
  \tightlist
  \item
    \emph{e.g.} for two binary variables \(\beta_1 = 1.5\),
    \(\beta_2 = 1.5\).
  \item
    If \(x_1=1\) and \(x_2=1\), this adds 3.0 to \(E(y|x)\)
  \end{itemize}
\item
  Logistic regression is a \emph{multiplicative} model

  \begin{itemize}
  \tightlist
  \item
    If \(x_1=1\) and \(x_2=1\), this adds 3.0 to \(log(\frac{P}{1-P})\)
  \item
    Odds-ratio \(\frac{P}{1-P}\) increases 20-fold: \(exp(1.5+1.5)\) or
    \(exp(1.5) * exp(1.5)\)
  \end{itemize}
\end{itemize}

\end{frame}

\hypertarget{motivating-example-for-log-linear-models}{%
\section{Motivating example for log-linear
models}\label{motivating-example-for-log-linear-models}}

\begin{frame}{Effectiveness of a depression case-management program}
\protect\hypertarget{effectiveness-of-a-depression-case-management-program}{}

\begin{itemize}
\tightlist
\item
  Research question: can a new treatment reduce the number of needed
  visits to the emergency room, compared to standard care?
\item
  \emph{outcome}: \# of emergency room visits for each patient in the
  year following initial treatment
\item
  \emph{predictors}:

  \begin{itemize}
  \tightlist
  \item
    \emph{race} (white or nonwhite)
  \item
    \emph{treatment} (treated or control)
  \item
    \emph{amount of alcohol consumption} (numerical measure)
  \item
    \emph{drug use} (numerical measure)
  \end{itemize}
\end{itemize}

\end{frame}

\begin{frame}{Statistical issues}
\protect\hypertarget{statistical-issues}{}

\begin{enumerate}
\tightlist
\item
  about 1/3 of observations are exactly 0 (did not return to the
  emergency room within the year)
\item
  highly nonnormal and cannot be transformed to be approximately normal
\item
  even \(log(y_i + 1)\) transformation will have a ``lump'' at zero +
  over 1/2 the transformed data would have values of 0 or \(log(2)\)
\item
  a linear regression model would give negative predictions for some
  covariate combinations
\item
  some subjects die or cannot be followed up on for a whole year
\end{enumerate}

\end{frame}

\hypertarget{poisson-log-linear-glm}{%
\section{Poisson log-linear GLM}\label{poisson-log-linear-glm}}

\begin{frame}{Towards a reasonable model}
\protect\hypertarget{towards-a-reasonable-model}{}

\begin{itemize}
\tightlist
\item
  A \emph{multiplicative} model will allow us to make inference on
  \emph{ratios} of mean emergency room usage
\item
  Modeling \(log\) of the \emph{mean} emergency usage ensures positive
  means, and does not suffer from \(log(0)\) problem
\item
  Random component of GLM, or residuals (was
  \(\epsilon_i \stackrel{iid}{\sim} N(0, \sigma_\epsilon^2)\) for linear
  regression) may still not be normal, but we can choose from other
  distributions
\end{itemize}

\end{frame}

\begin{frame}{Proposed model without time}
\protect\hypertarget{proposed-model-without-time}{}

\[
log(E[Y_i]) = \beta_0 + \beta_1 \textrm{RACE}_i + \beta_2 \textrm{TRT}_i + \beta_3 \textrm{ALCH}_i + \beta_4 \textrm{DRUG}_i
\] Or equivalently: \[
E[Y_i] = exp \left( \beta_0 + \beta_1 \textrm{RACE}_i + \beta_2 \textrm{TRT}_i + \beta_3 \textrm{ALCH}_i + \beta_4 \textrm{DRUG}_i \right)
\] where \(E[Y_i]\) is the expected number of emergency room visits for
patient \emph{i}.

\begin{itemize}
\tightlist
\item
  Important note: Modeling \(log(E[Y_i])\) is \emph{not} equivalent to
  modeling \(E(log(Y_i))\)
\end{itemize}

\end{frame}

\begin{frame}{Accounting for follow-up time}
\protect\hypertarget{accounting-for-follow-up-time}{}

Instead, model mean count per unit time: \[
\begin{aligned}
log(E[Y_i]/t_i) = \beta_0 + \beta_1 \textrm{RACE}_i + \beta_2 \textrm{TRT}_i + 
\beta_3 \textrm{ALCH}_i + \nonumber \\ \beta_4 \textrm{DRUG}_i
\end{aligned}
\]

Or equivalently: \[
\begin{aligned}
log(E[Y_i]) = \beta_0 + \beta_1 \textrm{RACE}_i + \beta_2 \textrm{TRT}_i + 
\beta_3 \textrm{ALCH}_i + \nonumber \\ \beta_4 \textrm{DRUG}_i + log(t_i)
\end{aligned}
\]

\begin{itemize}
\tightlist
\item
  \(log(t_i)\) is not a covariate, it is called an \emph{offset}
\end{itemize}

\end{frame}

\begin{frame}{The Poisson distribution}
\protect\hypertarget{the-poisson-distribution}{}

\begin{itemize}
\tightlist
\item
  Count data are often modeled as Poisson distributed:

  \begin{itemize}
  \tightlist
  \item
    mean \(\lambda\) is greater than 0
  \item
    variance is also \(\lambda\)
  \item
    Probability density
    \(P(k, \lambda) = \frac{\lambda^k}{k!} e^{-\lambda}\)
  \end{itemize}
\end{itemize}

\includegraphics{../docs/articles/session_lecture_files/figure-beamer/unnamed-chunk-1-1.pdf}

\end{frame}

\begin{frame}{When the Poisson distribution works}
\protect\hypertarget{when-the-poisson-distribution-works}{}

\begin{itemize}
\tightlist
\item
  Individual events are low-probability (small p), but many
  opportunities (large n)

  \begin{itemize}
  \tightlist
  \item
    e.g.~\# 911 calls per day
  \item
    e.g.~\# emergency room visits
  \end{itemize}
\item
  Approximates the binomial distribution when n is large and p is small

  \begin{itemize}
  \tightlist
  \item
    e.g.~\(n > 20\), \(np < 5\) or \(n(1-p) < 5\)
  \end{itemize}
\item
  When mean of residuals is approx. equal to variance
\end{itemize}

\end{frame}

\begin{frame}{GLM with log-linear link and Poisson error model}
\protect\hypertarget{glm-with-log-linear-link-and-poisson-error-model}{}

\begin{itemize}
\tightlist
\item
  Model the number of counts per unit time as Poisson-distributed + so
  the expected number of counts per time is \(\lambda_i\)
\end{itemize}

\(E[Y_i]/t_i = \lambda_i\) \newline \(log(E[Y_i]/t_i) = log(\lambda_i)\)
\newline \(log(E[Y_i]) = log(\lambda_i) + log(t_i)\) \newline

Recalling the log-linear model systematic component: \[
\begin{aligned}
log(E[Y_i]) = \beta_0 + \beta_1 \textrm{RACE}_i + \beta_2 \textrm{TRT}_i + 
\beta_3 \textrm{ALCH}_i + \nonumber \\ \beta_4 \textrm{DRUG}_i + log(t_i)
\end{aligned}
\]

\end{frame}

\begin{frame}{GLM with log-linear link and Poisson error model (cont'd)}
\protect\hypertarget{glm-with-log-linear-link-and-poisson-error-model-contd}{}

Then the systematic part of the GLM is: \[
log(\lambda_i) = \beta_0 + \beta_1 \textrm{RACE}_i + \beta_2 \textrm{TRT}_i + \beta_3 \textrm{ALCH}_i + \beta_4 \textrm{DRUG}_i
\] Or alternatively: \[
\lambda_i = exp \left( \beta_0 + \beta_1 \textrm{RACE}_i + \beta_2 \textrm{TRT}_i + \beta_3 \textrm{ALCH}_i + \beta_4 \textrm{DRUG}_i \right)
\]

\end{frame}

\begin{frame}{Interpretation of coefficients}
\protect\hypertarget{interpretation-of-coefficients}{}

\begin{itemize}
\tightlist
\item
  Suppose that \(\hat \beta_1 = -0.5\) in the fitted model, where
  \(\textrm{RACE}_i=0\) for white and \(\textrm{RACE}_i=1\) for
  non-white.
\item
  The mean rate of emergency room visits per unit time for white
  relative to non-white, all else held equal, is estimated to be:
\end{itemize}

\[
\frac{exp \left( \beta_0 + 0 + \beta_2 \textrm{TRT}_i + \beta_3 \textrm{ALCH}_i + \beta_4 \textrm{DRUG}_i \right)}{exp \left( \beta_0 - 0.5 + \beta_2 \textrm{TRT}_i + \beta_3 \textrm{ALCH}_i + \beta_4 \textrm{DRUG}_i \right)}
\] \[
= \frac{e^{\beta_0} e^0 e^{\beta_2 \textrm{TRT}_i} e^{\beta_3 \textrm{ALCH}_i} e^{\beta_4 \textrm{DRUG}_i}}
{e^{\beta_0} e^{-0.5} e^{\beta_2 \textrm{TRT}_i} e^{\beta_3 \textrm{ALCH}_i} e^{\beta_4 \textrm{DRUG}_i}}
\] \[
= \frac{e^0}{e^{-0.5}}
\] \[
= e^{0.5} \approxeq 1.65
\]

\end{frame}

\begin{frame}{Interpretation of coefficients (cont'd)}
\protect\hypertarget{interpretation-of-coefficients-contd}{}

\begin{itemize}
\tightlist
\item
  If \(\hat \beta_1=-0.5\) with whites as the reference group:

  \begin{itemize}
  \tightlist
  \item
    after adjustment for treatment group, alcohol and drug usage, whites
    tend to use the emergency room at a rate 1.65 times higher than
    non-whites.
  \item
    equivalently, the average rate of usage for whites is 65\% higher
    than that for non-whites
  \end{itemize}
\item
  Multiplicative rules apply for other coefficients as well, because
  they are exponentiated to estimate the mean rate.
\end{itemize}

\end{frame}

\hypertarget{multi-collinearity}{%
\section{Multi-collinearity}\label{multi-collinearity}}

\begin{frame}{What is Multicollinearity?}
\protect\hypertarget{what-is-multicollinearity}{}

\begin{enumerate}
\tightlist
\item
  \emph{Multicollinearity} exists when two or more of the independent
  variables in regression are moderately or highly correlated.
\item
  High correlation among continuous predictors or high concordance among
  categorical predictors
\item
  Impacts the ability to estimate regression coefficients

  \begin{itemize}
  \tightlist
  \item
    larger standard errors for regression coefficients
  \item
    ie, coefficients are unstable over repeated sampling
  \item
    exact collinearity produces infinite standard errors on coefficients
  \end{itemize}
\item
  Can also result in unstable (high variance) prediction models
\end{enumerate}

\end{frame}

\begin{frame}{Identifying multicollinearity}
\protect\hypertarget{identifying-multicollinearity}{}

\begin{enumerate}
\tightlist
\item
  Pairwise correlations of data or of model matrix (latter works with
  categorical variables)
\item
  Heat maps
\item
  Variance Inflation Factor (VIF) of regression coefficients
\end{enumerate}

\end{frame}

\begin{frame}[fragile]{Example: US Judge Ratings dataset}
\protect\hypertarget{example-us-judge-ratings-dataset}{}

See \texttt{?USJudgeRatings} for dataset, \texttt{?pairs} for plot code:
\includegraphics{../docs/articles/session_lecture_files/figure-beamer/unnamed-chunk-2-1.pdf}
**Pairwise scatterplot of continuous variables in US Judge Ratings
dataset

\end{frame}

\begin{frame}[fragile]{Example: iris dataset}
\protect\hypertarget{example-iris-dataset}{}

One categorical variable, so use model matrix. Make a simple heatmap.
\tiny

\begin{Shaded}
\begin{Highlighting}[]
\NormalTok{mm <-}\StringTok{ }\KeywordTok{model.matrix}\NormalTok{( }\OperatorTok{~}\StringTok{ }\NormalTok{., }\DataTypeTok{data =}\NormalTok{ iris)}
\NormalTok{pheatmap}\OperatorTok{::}\KeywordTok{pheatmap}\NormalTok{(}\KeywordTok{cor}\NormalTok{(mm[, }\DecValTok{-1}\NormalTok{]), }\CommentTok{#-1 gets rid of intercept column}
  \DataTypeTok{color =} \KeywordTok{colorRampPalette}\NormalTok{(}\KeywordTok{c}\NormalTok{(}\StringTok{"#f0f0f0"}\NormalTok{, }\StringTok{"#bdbdbd"}\NormalTok{, }\StringTok{"#636363"}\NormalTok{))(}\DecValTok{100}\NormalTok{))}
\end{Highlighting}
\end{Shaded}

\includegraphics{../docs/articles/session_lecture_files/figure-beamer/unnamed-chunk-3-1.pdf}
\emph{Note:} multicollinearity exists between multiple predictors, not
between predictor and outcome

\end{frame}

\begin{frame}[fragile]{Example: iris dataset}
\protect\hypertarget{example-iris-dataset-1}{}

Confirm what in iris dataset using Variance Inflation Factor of a linear
regression model: \tiny

\begin{Shaded}
\begin{Highlighting}[]
\NormalTok{fit <-}\StringTok{ }\KeywordTok{lm}\NormalTok{(Sepal.Width }\OperatorTok{~}\StringTok{ }\NormalTok{., }\DataTypeTok{data =}\NormalTok{ iris)}
\NormalTok{car}\OperatorTok{::}\KeywordTok{vif}\NormalTok{(fit)}
\end{Highlighting}
\end{Shaded}

\begin{verbatim}
##                   GVIF Df GVIF^(1/(2*Df))
## Sepal.Length  6.124653  1        2.474804
## Petal.Length 45.132550  1        6.718076
## Petal.Width  18.373804  1        4.286468
## Species      32.701564  2        2.391344
\end{verbatim}

\end{frame}

\begin{frame}{Approaches for dealing with multicollinearity}
\protect\hypertarget{approaches-for-dealing-with-multicollinearity}{}

Options:

\begin{enumerate}
\tightlist
\item
  Select a representative variable
\item
  Average variables
\item
  Principal Component Analysis or other dimension reducuction
\item
  For prediction modeling, special methods like penalized regression,
  Support Vector Machines, \ldots{}
\end{enumerate}

\end{frame}

\hypertarget{conclusions}{%
\section{Conclusions}\label{conclusions}}

\begin{frame}{Conclusions}
\protect\hypertarget{conclusions-1}{}

\begin{enumerate}
\tightlist
\item
  Log-linear models are appropriate for non-negative, skewed count data

  \begin{itemize}
  \tightlist
  \item
    probability of each event is low
  \end{itemize}
\item
  The coefficients of log-linear models are \emph{multiplicative}
\item
  An \emph{offset} term can account for varying follow-up time or
  otherwise varying opportunity to be counted
\item
  Poisson distribution is limit of binomial distribution with high
  number of trials, low probability
\item
  Inference from log-linear models is sensitive to the choice of error
  model (assumption on the distribution of residuals)
\item
  We will cover other options next week for when the Poisson error model
  doesn't fit:

  \begin{itemize}
  \tightlist
  \item
    Variance proportional to mean, instead of equal
  \item
    Negative Binomial
  \item
    Zero Inflation
  \end{itemize}
\end{enumerate}

\end{frame}

\end{document}
